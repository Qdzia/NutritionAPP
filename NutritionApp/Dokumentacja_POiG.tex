\documentclass[12pt,a4paper]{article}
\usepackage[polish]{babel}
\usepackage[T1]{fontenc}
\usepackage[utf8x]{inputenc}
\usepackage{hyperref}
\usepackage{url}
\usepackage[]{algorithm2e}
\usepackage{listings}

\usepackage{color}
\usepackage{listings}

\lstloadlanguages{% Check Dokumentation for further languages ...
	C,
	C++,
	csh,
	Java
}

\definecolor{red}{rgb}{0.6,0,0} % for strings
\definecolor{blue}{rgb}{0,0,0.6}
\definecolor{green}{rgb}{0,0.8,0}
\definecolor{cyan}{rgb}{0.0,0.6,0.6}

\lstset{
	language=csh,
	basicstyle=\footnotesize\ttfamily,
	numbers=left,
	numberstyle=\tiny,
	numbersep=5pt,
	tabsize=2,
	extendedchars=true,
	breaklines=true,
	frame=b,
	stringstyle=\color{blue}\ttfamily,
	showspaces=false,
	showtabs=false,
	xleftmargin=17pt,
	framexleftmargin=17pt,
	framexrightmargin=5pt,
	framexbottommargin=4pt,
	commentstyle=\color{green},
	morecomment=[l]{//}, %use comment-line-style!
	morecomment=[s]{/*}{*/}, %for multiline comments
	showstringspaces=false,
	morekeywords={ abstract, event, new, struct,
		as, explicit, null, switch,
		base, extern, object, this,
		bool, false, operator, throw,
		break, finally, out, true,
		byte, fixed, override, try,
		case, float, params, typeof,
		catch, for, private, uint,
		char, foreach, protected, ulong,
		checked, goto, public, unchecked,
		class, if, readonly, unsafe,
		const, implicit, ref, ushort,
		continue, in, return, using,
		decimal, int, sbyte, virtual,
		default, interface, sealed, volatile,
		delegate, internal, short, void,
		do, is, sizeof, while,
		double, lock, stackalloc,
		else, long, static,
		enum, namespace, string},
	keywordstyle=\color{cyan},
	identifierstyle=\color{red},
}
\usepackage{caption}
\DeclareCaptionFont{white}{\color{white}}
\DeclareCaptionFormat{listing}{\colorbox{blue}{\parbox{\textwidth}{\hspace{15pt}#1#2#3}}}
\captionsetup[lstlisting]{format=listing,labelfont=white,textfont=white, singlelinecheck=false, margin=0pt, font={bf,footnotesize}}


\addtolength{\hoffset}{-1.5cm}
\addtolength{\marginparwidth}{-1.5cm}
\addtolength{\textwidth}{3cm}
\addtolength{\voffset}{-1cm}
\addtolength{\textheight}{2.5cm}
\setlength{\topmargin}{0cm}
\setlength{\headheight}{0cm}

\begin{document}
	
	\title{Programowanie obiektowe i graficzne\\\small{dokumentacja projektu NutritionApp}}
	\author{Dariusz Momot \\ Łukasz Kudzia \\ grupa 2E}
	\date{\today}

	\maketitle
	\newpage
	\section*{Część I}
	\subsection*{Opis programu}
	NurtritionApp to narzędzie pomagające w planowaniu posiłków oraz listy zakupów na dany tydzień.
	Program posiada podstawową listę przepisów, którą można modyfikować poprzez dodawanie lub usuwanie wybranego przepisu.
	Podczas planowania posiłków na dany tydzień można skorzystać z już stworzonych przepisów.
	Przy tak stworzonym planie w zakładce \textit{Grocery List} możemy zobaczyć listę zakupów potrzebnych na wybrane posiłki,
	w której możemy usunąć produkty już posiadane w lodówcę a następnie zapisać listę zakupów w pliku o rozszerzeniu \textit{pdf}.
	
	\subsection*{Instrukcja obsługi}
	Jak uruchomić program, jak wyglądają dane. Mile widziana wizualizacja gry, wyników z punku widzenia aplikacji itd. 
	\subsection*{Dodatkowe informacje}
	Wymagania itd.
	\newpage
	\section*{Część II}
	\subsection*{Opis działania} 
	Tutaj uwzględniamy część matematyczną. Opisujemy całą teorię np.:
	dla zadania związanego z sieciami neuronowymi - opisujemy całą budowę, algorytm uczenia i wszystkie wzory. Dla zadania związanego z kombinatoryką opisujemy całą teorię kombinatoryczną potrzebną do zrozumienia zadania (mile widziany przykład obliczeniowy).
	
	
	
	\subsection*{Algorytm}
	Tutaj opisujemy rozwiązanie zadania. Dla przedmiotu programowanie będzie to wykorzystanie matematyki z poprzedniego zadania itd. Dla SSI będzie to ogólne działanie przetwarzania danych w oparciu o modele matematyczne z poprzedniego zadania. 
	
	
	Pseudokod tworzymy w \LaTeX. Przykład:\\
	\begin{algorithm}[H]
		\KwData{Dane wejściowe liczba $k$}
		\KwResult{Brak }
		$i:=0$\;
		\While{$i<k$}{
			Drukuj na ekran liczbę $i$\;
			\eIf{$i\%2==0$}{
				Wydrukj informację, że liczba $i$ jest liczbą parzystą\;
			}{
				Wydrukj informację, że liczba $i$ nie jest liczbą parzystą\;
			}
		}
		\caption{Algorytm drukowania informacji o liczbie parzystej/nieprarzystej.}
	\end{algorithm}

	\subsection*{Bazy danych}
	Sekcja wystepuje tylko w przypadku projektów bazodanowych.
	
	Należy pokazać przykładowe dane, które były wykorzystywane podczas uczenia klasyfikatorów.
	
	Strukturę bazy i relacje.
	\subsection*{Implementacja}
	Opis, zasada i działanie programu ze względu na podział na pliki, nastepnie	funkcje programu wraz ze szczegółowym opisem działania (np.: formie pseudokodu, czy odniesienia do równania)
	\begin{lstlisting}
	Tutaj wklejamy fragment kodu, ktory chcemy opisac 
	(bez polskich znakow).
	\end{lstlisting}
	\subsection*{Testy}
	Tutaj powinna pojawić się analiza uzyskanych wyników oraz wykresy/pomiary.
	
	\subsection*{Eksperymenty}
	Sekcję używamy gdy porównywaliśmy dwa lub więcej algorytmów, albo wykonywaliśmy jakies pomiery.
	
	Warto dodać jakies wykresy jako obraz, albo tabele z wynikami. 
	
	Wszyskie wyniki powinny być opisane/poddane komentarzowi i poddane analizie statystycznej.
	\newpage
	\section*{Pełen kod aplikacji}
\begin{lstlisting}
Tutaj wklejamy pelen kod. 
\end{lstlisting}
\end{document}
